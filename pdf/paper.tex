\documentclass[a4paper, 12pt]{extarticle}

% Import packages
%------------------------------------------------
\usepackage[utf8]{inputenc}  % utf8 encoding
\usepackage{csquotes}        % for Russian quotes
\usepackage[russian]{babel}  % enable Russian language
\usepackage{titling}         % for \thetitle, \thedate
\usepackage{enumitem}        % for list customization
\usepackage{graphicx}        % for adding images
\usepackage{prettyref}       % for configuring references
\usepackage{caption}
\usepackage{wrapfig}
\usepackage{enumitem}        % for lists customization
\usepackage{graphicx}        % for adding images
\usepackage{prettyref}       % for configuring references
\usepackage{caption}         % for figure caption customization
\usepackage{subcaption}      % for \subfigure
\usepackage{wrapfig}         % for figures and images

\usepackage{tikz}            % for diagrams
\usetikzlibrary{arrows,shapes,positioning,shadows,trees}
\tikzstyle{every node}=[draw=black,thin,anchor=west, minimum height=2.5em]

%------------------------------------------------
% Geometry
%------------------------------------------------
\usepackage{geometry}
\geometry{
    top=1.5cm,        % Top margin
    bottom=1.5cm,     % Bottom margin
    left=2cm,         % Left margin
    right=2cm,        % Right margin
    includefoot,      % Include space for a footer
}

%------------------------------------------------
% Table of Context
%------------------------------------------------
\usepackage[titles]{tocloft}                      % style of table of context
\usepackage{hyperref}                             % clickable table of context items
\setcounter{tocdepth}{2}                          % set ToC depth. (2: sections+subsubsections)
\renewcommand{\cftsecafterpnum}{\vspace{10pt}}    % set space after section
\renewcommand{\cftsubsecafterpnum}{\vspace{5pt}}  % set space after subsection
\renewcommand{\cftsecfont}{\bfseries \large}      % change section font

\hypersetup{
    colorlinks,
    citecolor=black,
    filecolor=black,
    linkcolor=black,
    urlcolor=black
}

%------------------------------------------------
% Sections style
%------------------------------------------------
\usepackage{titlesec} % for changing style of sections

% Section settings
\titleformat{\section}
{\bfseries\LARGE}
{\thesection. }
{4pt}
{}
[{\titlerule[0.4pt]}]

% Subsection settings
\titleformat{\subsection}
{\bfseries\Large}
{\thesubsection. }
{0pt}
{}

% Subsubsection settings
\titleformat{\subsubsection}
{\bfseries\large}
{}
{0pt}
{}

%------------------------------------------------
% Other settings
%------------------------------------------------
\setlength{\parindent}{0pt}
\setlist{itemsep=1pt}
\newrefformat{fig}{(рис. \ref{#1})}
\addto{\captionsrussian}{
  \renewcommand{\refname}{Список использованных источников}
}

%------------------------------------------------
% Diagrams settings
%------------------------------------------------
%------------------------------------------------
% Title, author, date
%------------------------------------------------
\title{Актуализация информатики путем непосредственной разработки сайта,
позволяющего ученикам 11-х классов эффективно подготовиться к ЕГЭ}
\author{Меркулов Вадим}
\selectlanguage{russian}
\date{\today}
%------------------------------------------------

\begin{document}
\begin{titlepage}
    \vspace*{0.7cm}
    \begin{center}
        \begin{figure}[h]
            \includegraphics{../img/title_page_school_logo.jpg}
        \end{figure}
        \vspace{1.5cm}

        \Large{\textbf{
            Индивидуальный проект \\
            \textquote{\thetitle}}
        }
    \end{center}
    \vspace{5cm}

    \begin{flushright}
        Работу выполнил:\\ \vspace{7pt}
        Ученик 10 \textquote{А} класса\\ \vspace{7pt}
        \theauthor\\ \vspace{1cm}
        Научный руководитель:\\ \vspace{7pt}
    \end{flushright}
    \vfill
    \centering
    Москва, {\the\year} г.

\end{titlepage}

\tableofcontents
\newpage

\section{Введение}

\vspace{2mm}
\textbf{Актульность}
\vspace{2mm}
\\
Ни для кого не секрет, что сдача Единого Государственного Экзамена вместе с
подготовкой к ней становится значимым периодом в жизни для большей части
учеников старшей школы, заинтересованных в своем будущем. В связи с этим у
многих подростков невольно возникает вопрос, как именно стоит готовиться к
столь важному испытанию, чтобы затраченные силы были прямо пропорциональны
полученным результатам, а время подготовки впоследствии не стало предметом
сожаления?  В век переизбытка цифровой информации становится все труднее
абстрагироваться от ненужного и сфокусировать все свои усилия на одном
предмете, не распыляясь на множество задач. На мой взгляд именно сейчас
необходимо и по-настоящему актуально создать ресурс, который сможет
предоставить ученику широкий спектр заданий для выполнения, будет простым и
понятным в использовании и содержащий только необходимые материалы, напрямую относящиеся к подготовке к ЕГЭ.
\\

\vspace{2mm}
\textbf{Цель}
\vspace{2mm}
\\
Внедрение информатики в процесс подготовки к
сдаче ЕГЭ и его облегчение путем непосредственной разработки сайта.
\\

\vspace{2mm}
\textbf{Объект}
\vspace{2mm}
\\
Сниженная эффективность учащихся при самостоятельной подготовке к ЕГЭ.
\\

\vspace{2mm}
\textbf{Предмет}
\vspace{2mm}
\\
Способ повышения эффективности обучения учеников в интернете
путем разработки собственного ресурса по подготовке к экзамену.
\\

\vspace{2mm}
\textbf{Гипотеза}
\vspace{2mm}
\\
Информатика может облегчить процесс тренировки и подготовки
старшеклассников к сдаче ЕГЭ.
\\

\vspace{2mm}
\textbf{Задачи:}
\begin{itemize}
    \item[\bfseries--] Создать интуитивно понятный сайт по подготовке к ЕГЭ
    \item[\bfseries--] Исключить ненужные данные, которые могут отвлечь
        учащегося от выполнения заданий
    \item[\bfseries--] Сделать ресурс воспроизводимым как с компьютеров, так и
        с мобильных устройств
\end{itemize}

\vspace{2mm}
\textbf{Методы:}
\begin{itemize}
    \item[\bfseries--] Анализ информационных порталов с похожей специализацией
    \item[\bfseries--] Анализ потенциальных причин отвлечения внимания во время
        самоподготовки
    \item[\bfseries--] Опора на собственный опыт в подготовке к экзаменам, а
        также на опыт сверстников
\end{itemize}
\newpage

\section{Основные сведения о подготовке к Единому Государственному Экзамену}
\subsection{Современные методы самоподготовки к экзамену}
\vspace{2mm}
Современному школьнику не составляет никакого труда найти информацию по
интересующей его теме с помощью Интернета. Открытые ресурсы в разы облегчают
поиск нужных данных и снижают время, потраченное на него. Для тех, кто ищет, в век
технологий открыты все двери, и каждый день появляются все новые и новые
возможности. Каждому ученику старшей школы уже давно были известны такие
порталы, как
\begin{itemize}
    \item {\small Яндекс Репетитор\par}
    \item {\small Решу ЕГЭ\par}
    \item {\small Мои достижения\par}
    \item {\small Библиотека МЭШ\par}
\end{itemize}
Именно они содержат самый широкий спектр предметов и заданий по ним, что
приносит значительную пользу как учащимся при подготовке, так и преподавателям, ведь собирать
однотипные задачи из разных источников — задача не из легких. Также при изучении
новой темы и при закреплении пройденного часто приходят на выручку видео на
YouTube с более детальным разбором темы и живым голосовым сопровождением.
\\

Существует ли какая-либо альтернатива у подобных методов обучения?
Безусловно, она есть. Сфера подготовки к экзаменам полна различных вариантов
и возможностей, все зависит лишь от ресурсов и желаний самих учеников.
Помимо открытых и бесплатных ресурсов, перечисленных выше, существуют также
немалоизвестные онлайн школы, например,
\begin{itemize}
    \item {\small Фоксфорд\par}
    \item {\small Умскул\par}
    \item {\small Вебинариум\par}
\end{itemize}
Те, кому важна портативность и возможность учиться в удобное время, а
также те, кому необязательно присутствие наставника рядом, часто выбирают именно этот
метод обучения, поскольку он устанавливает определенные рамки сдачи выполненных
заданий, что подталкивает учеников к развитию.
\\

Каждый школьник избирает путь подготовки, удобный лично ему. Так, например,
некоторые ученики не могут справиться с изучением и закреплением материала без
внешней поддержки и непосредственного присутствия преподавателя, готового помочь
ему с этим. Некоторым же, наоборот, хватает пары видеоразборов по
интересующему вопросу и открытых банков заданий для тренировки. К счастью,
каждому есть из чего выбрать и найти метод, удовлетворяющий ученика не только
по материальным причинам, но также, что самое главное, по своей эффективности.
В дальнейшем более детально будет рассматриваться именно бесплатный метод
подготовки.
\newpage

\subsection{Почему мы отвлекаемся?}
Проблема, знакомая многим школьникам с ранних лет, — это неспособность
сохранять сосредоточенность. Стоит заметить, что рассеянность внимания
стала настолько распространенным препятствием, встающим на пути между
школьниками и знаниями, что начала входить в норму для некоторых, перестала
быть заметной и начала забирать намного больше времени, у каждого, кто не умеет
с ней бороться. Неустойчивость внимания характеризуется тем, что человек не способен
сосредоточиться на деле, которое требует долгосрочной перспективы или
длительной концентрации. Это и является обратной стороной богатого выбора
информационных ресурсов.
\\

Современной молодежи зачастую становится трудно, а порой и вовсе невозможно
отказаться от потребления ненужной информации в Интернете. Часто сами ресурсы,
побуждают ученика перевести его внимание с учебы на что-либо еще, не связанное
с темой образования. Распространенные причины подобного поведения:
\begin{itemize}
    \item[\bfseries--] {\small Реклама\par}
    \item[\bfseries--] {\small Уведомления\par}
    \item[\bfseries--] {\small Перегруженный интерфейс\par}
\end{itemize}
Наверняка каждый испытывал раздражение от навязчивой рекламы или от
загруженности веб-страницы, найти нужную информацию на которой можно только
после долгих поисков.
\\

На мой взгляд первостепенной проблемой в наше время является тот факт, что
теперь многие просто перестают замечать, сколько времени тратится по причине
информационной перенасыщенности. Сегодня достаточно отвлечься на единственное
всплывающее окно, чтобы выйти из состояния потока, оставить прежнее занятие,
тратя все больше и больше времени, переходя на все новые сайты. Следовательно,
неумение правильно организовать свое рабочее пространство и время, оградившись от всего
ненужного и отвлекающего, стало значительным затруднением для современной
молодежи.
\\

Чтобы избежать рассеянности внимания во время рабочего процесса самим учащимся
стоит придерживаться некоторых правил. Вот некоторые из них:
\begin{itemize}
    \item[\bfseries--] {\small Избегать переутомления\par}
    \item[\bfseries--] {\small Хорошо высыпаться\par}
    \item[\bfseries--] {\small Уметь расставлять приоритеты\par}
    \item[\bfseries--] {\small Ставить перед собой реальные задачи\par}
\end{itemize}
Однако, многое также зависит от обучающих веб-ресурсов. С моей точки зрения
подобные сайты не должны провоцировать студента заняться чем-то кроме учебы,
напротив, их основным приоритетом должна быть эффективность передачи знаний.
Чем меньше требуется усилий, чтобы приступить к работе на сайте, чем понятнее и
проще он выглядит, тем, на мой взгляд, выше продуктивность такой деятельности.
\newpage

\subsection{Решение проблемы}

\newpage

\end{document}
